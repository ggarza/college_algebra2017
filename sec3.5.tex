\documentclass[oneside,10pt]{book}
\usepackage{graphicx}
\graphicspath{{pictures/}}% Specifies where pictures are stored
\usepackage{docmute} % for muting preamble of input files
\usepackage[top=0.65in,bottom=0.50in,left=0.75in,right=0.75in]{geometry}
\usepackage{amsmath}
\usepackage{amssymb}
\usepackage{multicol}
\usepackage{booktabs}
\usepackage[shortlabels]{enumitem}

\usepackage[activate={true,nocompatibility},final,tracking=true,kerning=true,spacing=true,factor=1100,stretch=10,shrink=10]{microtype}

\usepackage{subcaption} % for \captionof command
\usepackage{mathpazo} % Times
\usepackage[T1]{fontenc}
\usepackage[semibold]{raleway}

\usepackage[usenames,dvipsnames,svgnames]{xcolor} % for colors
\definecolor{ocre}{RGB}{0,135,255} % main color
\definecolor{maincolor}{RGB}{0,145,255} % main color
\definecolor{cqcqcq}{rgb}{0.75294117,0.75294117647,0.75294117647}

\usepackage{tikz}
\usetikzlibrary{arrows.meta, arrows}

\usepackage{xargs}
\usepackage{colortbl} % for colored tables
\usepackage{hyperref}
%\hypersetup{
    %colorlinks=true,
    %linkcolor=blue,
    %filecolor=magenta,
    %urlcolor=cyan,
%}





\makeatletter
%%%%%%%%%%%%%%%%%%%%%%%%%%%%%%%%%%%%%%%%%%%%%%%%%%%%%%%%%%%%%%%%%%%
% Set indent to zero, but save value just in case
%%%%%%%%%%%%%%%%%%%%%%%%%%%%%%%%%%%%%%%%%%%%%%%%%%%%%%%%%%%%%%%%%%%
\newlength\tindent
\setlength{\tindent}{\parindent}
\setlength{\parindent}{0pt}
\renewcommand{\indent}{\hspace*{\tindent}}
%%%%%%%%%%%%%%%%%%%%%%%%%%%%%%%%%%%%%%%%%%%%%%%%%%%%%%%%%%%%%%%%%%%
%%%%%%%%%%%%%%%%%%%%%%%%%%%%%%%%%%%%%%%%%%%%%%%%%%%%%%%%%%%%%%%%%%%

\newcounter{ExampleCounter}[section]
\newcommand{\example}{%
  \vspace{4pt minus 3pt}
  \par%
  \refstepcounter{ExampleCounter}%
  \noindent\textbf{Example~\arabic{ExampleCounter}.\,~}%
}



%%%%%%%%%%%%%%%%%%%%%%%%%%%%%%%%%%%%%%%%%%%%%%%%%%%%%%%%%%%%%%%%%%%
% grid command
%%%%%%%%%%%%%%%%%%%%%%%%%%%%%%%%%%%%%%%%%%%%%%%%%%%%%%%%%%%%%%%%%%%
\newcommand{\grid}[1]{%
\begin{tikzpicture}
  \draw[step=.5cm,cqcqcq,very thin] (-#1/2,-#1/2) grid (#1/2,#1/2);
\end{tikzpicture}
}


\newcommandx{\cart}[3][2=1,3={}]{%
\par
\begin{tikzpicture}[scale=#2]
  \draw[step=.5cm,cqcqcq,very thin] (-#1/2,-#1/2) grid (#1/2,#1/2);
  \draw[<->, black] (-#1/2,0) -- (#1/2,0);
  \draw[<->, black] (0, -#1/2) -- (0, #1/2);
  #3
\end{tikzpicture}
\par
}


\newcommandx{\cartX}[6][5=1,6={}]{%
\par
\begin{tikzpicture}[scale=#5]
  \draw[step=1cm,cqcqcq,very thin] (#1,#3) grid (#2,#4);
  \draw[<->, black] (#1,0) -- (#2,0);
  \draw[<->, black] (0, #3) -- (0, #4);
  #6
\end{tikzpicture}
\par
}
%%%%%%%%%%%%%%%%%%%%%%%%%%%%%%%%%%%%%%%%%%%%%%%%%%%%%%%%%%%%%%%%%%%
%%%%%%%%%%%%%%%%%%%%%%%%%%%%%%%%%%%%%%%%%%%%%%%%%%%%%%%%%%%%%%%%%%%


%%%%%%%%%%%%%%%%%%%%%%%%%%%%%%%%%%%%%%%%%%%%%%%%%%%%%%%%%%%%%%%%%%%%%%%%%%%
% Headings
%%%%%%%%%%%%%%%%%%%%%%%%%%%%%%%%%%%%%%%%%%%%%%%%%%%%%%%%%%%%%%%%%%%%%%%%%%%


\usepackage{titlesec}

\titleformat{\chapter}[display]
  {\sffamily\huge\bfseries}{\color{maincolor}
  {\chaptertitlename\ \thechapter}}{0pt}{\Huge}[]
\titleformat{\section}
  {\sffamily\LARGE\bfseries}
  {\color{maincolor}{\thesection}}{8pt}{}[]
\titleformat{\subsection}
  {\sffamily\Large\bfseries}{}{0ex}{}
\titleformat{\subsubsection}
  {\sffamily\normalsize\bfseries}{\thesubsubsection}{1ex}{}
\titleformat{\paragraph}[runin]
  {\sffamily\normalsize\bfseries}{\theparagraph}{1em}{}
\titleformat{\subparagraph}
  {\sffamily\LARGE\bfseries}
  {\color{maincolor}{\thesection}}{8pt}{}[]

\titlespacing*{\chapter} {0in}{0pt}{9pt}
\titlespacing*{\section} {0in}{3.5ex plus 1ex minus .2ex}{3.3ex plus .2ex}
\titlespacing*{\subsection} {0pt}{3.25ex plus 1ex minus .2ex}{1.5ex plus .2ex}
\titlespacing*{\subsubsection}{0pt}{3.25ex plus 1ex minus .2ex}{0.5ex plus .2ex}
\titlespacing*{\paragraph} {0pt}{3.25ex plus 1ex minus .2ex}{1em}
%\titlespacing*{\subparagraph} {\parindent}{3.25ex plus 1ex minus .2ex}{1em}
\titlespacing*{\subparagraph} {-0in}{3.5ex plus 1ex minus .2ex}{2.3ex plus .2ex}

\renewcommand{\bottomtitlespace}{2.5in}
%\newcommand{\sectionbreak}{\clearpage}
%\newcommand{\chapterbreak}{\cleardoublepage}


%%%%%%%%%%%%%%%%%%%%%%%%%%%%%%%%%%%%%%%%%%%%%%%%%%%%%%%%%%%%%%%%%%%%%%
%%%%%%%%%%%%%%%%%%%%%%%%%%%%%%%%%%%%%%%%%%%%%%%%%%%%%%%%%%%%%%%%%%%%%%

\newcommand{\setsection}[2]{%
  \setcounter{chapter}{#1}
  \setcounter{section}{#2}
  \addtocounter{section}{-1}

}

%%%%%%%%%%%%%%%%%%%%%%%%%%%%%%%%%%%%%%%%%%%%%%%%%%%%%%%%%%%%%%%%%%%%%%%%%%%
% Table of Contents Styling
%%%%%%%%%%%%%%%%%%%%%%%%%%%%%%%%%%%%%%%%%%%%%%%%%%%%%%%%%%%%%%%%%%%%%%%%%%%


\usepackage{titletoc} % Required for manipulating the table of contents
\contentsmargin{0cm} % Removes the default margin


% Part text styling
\titlecontents{part}[0cm]
{\addvspace{20pt}\centering\large\bfseries}
{}
{}
{}

% Chapter text styling
\titlecontents{chapter}[1.00cm] % Indentation
{\addvspace{5pt}\Large\sffamily\bfseries} % Spacing and font options for chapters
{\color{maincolor}\contentslabel[\Large\thecontentslabel]{1.00cm}\color{maincolor}} % Chapter number
{\color{maincolor}}
{\color{maincolor}\Large\;\titlerule*[.5pc]{.}\;\thecontentspage} % Page number

% Section text styling
\titlecontents{section}[1.75em] % Indentation
{\addvspace{5pt}\large\sffamily\bfseries} % Spacing and font options for chapters
{\color{maincolor}\contentslabel[\large\thecontentslabel]{1.75em}\color{black}} % Chapter number
{\color{black}}
{\color{black}\large\;\titlerule*[.5pc]{.}\;\thecontentspage} % Page number
[]

% Subsection text styling
\titlecontents{subsection}[1.75em] % Indentation
{\addvspace{0pt}\sffamily\small} % Spacing and font options for subsections
{} % Subsection number
{}
{\ \titlerule*[.5pc]{.}\;\thecontentspage} % Page number
[]


%%%%%%%%%%%%%%%%%%%%%%%%%%%%%%%%%%%%%%%%%%%%%%%%%%%%%%%%%%%%%%%%%%%%%%%%%%%
% PAGE HEADERS
%%%%%%%%%%%%%%%%%%%%%%%%%%%%%%%%%%%%%%%%%%%%%%%%%%%%%%%%%%%%%%%%%%%%%%%%%%%


\usepackage{ifthen}
\usepackage{fancyhdr} % Required for header and footer configuration

\pagestyle{fancy}

\renewcommand{\chaptermark}[1]{\markboth{\normalsize\bfseries\chaptername\ \thechapter\ \ #1}{}} % Chapter text font settings

\renewcommand{\sectionmark}[1]{\markright{\normalsize\thesection\hspace{5pt}#1}{}} % Section text font settings
\fancyhf{}


\fancyhead[R]{\sffamily\rightmark\hspace{5ex}\thepage}
%\fancyhead[L]{\ifthenelse{\isodd{\value{page}}}{}{\sffamily\thepage\hspace{5ex} \leftmark}}


\renewcommand{\headrulewidth}{0pt}
\renewcommand{\footrulewidth}{0pt} % Removes the rule in the footer

\addtolength{\headheight}{2.5pt} % Increase the spacing around the header slightly

%%%%%%%%%%%%%%%%%%%%%%%%%%%%%%%%%%%%%%%%%%%%%%%%%%%%%
%%%%%%%%%%%%%%%%%%%%%%%%%%%%%%%%%%%%%%%%%%%%%%%%%%%%%
% define a command to insert a blank page
        \newcommand{\insertblankpage}{%
          \newpage
          \thispagestyle{empty}
          \mbox{}
        \addtocounter{page}{-1}
          \newpage
        }
%%%%%%%%%%%%%%%%%%%%%%%%%%%%%%%%%%%%%%%%%%%%%%%%%%%%%
%%%%%%%%%%%%%%%%%%%%%%%%%%%%%%%%%%%%%%%%%%%%%%%%%%%%%



%%%%%%%%%%%%%%%%%%%%%%%%%%%%%%%%%%%%%%%%%%%%%%%%%%%%%
% Define Objectives Environment
%%%%%%%%%%%%%%%%%%%%%%%%%%%%%%%%%%%%%%%%%%%%%%%%%%%%%
\usepackage{environ}

\NewEnviron{objectives}[1]{%
\vspace{0.5em}
\noindent\textbf{\sffamily\Large Objectives}

\vspace{2mm}

\noindent%
#1

\vspace{-2.5mm}
\begin{multicols}{2}
\begin{itemize}[nosep,leftmargin=10.0pt]
    \BODY
\end{itemize}
\end{multicols}

}


%%%%%%%%%%%%%%%%%%%%%%%%%%%%%%%%%%%%%%%%%%%%%%%%%%%%%%%%%%%%%%%%%%%
% Create blue box for definitions and theorems
%%%%%%%%%%%%%%%%%%%%%%%%%%%%%%%%%%%%%%%%%%%%%%%%%%%%%%%%%%%%%%%%%%%


\tikzstyle{bluebox} = [fill=ocre!10, draw=ocre, very thick,
    rectangle, rounded corners, inner sep=10pt, inner ysep=5pt]
\tikzstyle{fancytitle} =[fill=ocre, text=white, rounded corners,
    inner ysep=3pt, inner xsep=5pt]



\def\blueboxstring{bluebox}


\newcommand{\bluebox}[2]{%
\par
\vspace{2mm minus 1mm}
\begin{tikzpicture}%
\node [bluebox] (box){%
    \begin{minipage}{0.99\textwidth}%
      \vspace{7pt}
      #2
    \end{minipage}
};
\node[fancytitle, right=10pt] at (box.north west) {%
  #1
};
\end{tikzpicture}
\par
\vspace{2mm}
}


%\NewEnviron{bluebox}[1]{%
%\par
%\vspace{2mm minus 1mm}
%\begin{tikzpicture}%
%\node [bluebox] (box){%
    %\begin{minipage}{0.99\textwidth}%
      %\vspace{7pt}
      %\expandafter\BODY
    %\end{minipage}
%};
%\node[fancytitle, right=10pt] at (box.north west) {%
  %#1
%};
%\end{tikzpicture}
%\par
%\vspace{2mm}
%}


%\let\blueboxenvironment\bluebox


%\def\bluebox#1{%
  %\ifx\@currenvir\blueboxstring%
    %\blueboxenvironment
  %\else
    %\blueboxcommand
  %\fi
%}



%%%%%%%%%%%%%%%%%%%%%%%%%%%%%%%%%%%%%%%%%%%%%%%%%%%%%
% Redefine vfill command to take optional parameter
%%%%%%%%%%%%%%%%%%%%%%%%%%%%%%%%%%%%%%%%%%%%%%%%%%%%%

\renewcommand{\vfill}[1][1]{\vspace{\stretch{#1}}}

\makeatother


\usepackage{booktabs}



\AtBeginDocument{%
  \large
}

\begin{document}

\setsection{3}{5}
\section{Graphing Transformations of Functions}


\textit{Sometimes the questions are complicated and the answers are simple.}
--Dr.~Suess


\begin{objectives}
{In this section we will}
  \item
    graph functions using Vertical and Horizontal shifts.
  \columnbreak
  \item
    graph functions using Compressions and Stretches.
  \item
    graph functions using reflections.

\end{objectives}






In this section you will need a laptop or tablet. We will be using an online
graphing program called \href{https://www.desmos.com/calculator}{Desmos}. Go to
\url{https://www.desmos.com/calculator}.

\vspace{5mm}

\bluebox{Graphing Techniques}{%

\begin{tabular}{@{}p{0.30\linewidth}p{0.65\linewidth}}
   \textbf{To graph}:  & \textbf{Draw the graph of $f$ and:} \\
   \midrule
  \textbf{Vertical Shifts} \\
  $g(x) = f(x) + k$,\;\; $k >0$ & Raise the graph of $f$  by $k$ units. \\
  $g(x) = f(x) - k$,\;\; $k >0$ & Lower the graph of $f$  by $k$ units.\\
  \\
  \textbf{Horizontal Shifts} \\
  $g(x) = f(x+h)$,\;\; $h >0$ & Shift the graph of $f$ to the left by $h$ units. \\
  $g(x) = f(x-h)$,\;\; $h >0$ & Shift the graph of $f$ to the right by $h$ units.\\
  \\
  \textbf{Compression or Stretching} \\
  $g(x) = af(x)$,\;\; $a >0$ & Stretch/shrink the graph of $f$ vertically by a
  factor of $a$. \\
  $g(x) = f(ax)$,\;\; $a >0$ & Stretch/shrink the graph of $f$ horizontally by a
  factor of $1/a$. \\
  \\
  \textbf{Reflection about the $x$-axis} \\
  $g(x) = -f(x)$ & Reflect the graph of $f$ about the $x$-axis (horizontal axis).\\
  \\
  \textbf{Reflection about the $y$-axis} \\
  $g(x) = f(-x)$ & Reflect the graph of $f$ about the $y$-axis (vertical axis).
\end{tabular}

}


\subsection{Vertical Shift}


\example
The graph of $f(x)=x^2$ is given below.  Fill in the table and graph the
function $g(x) = x^2 + 3$.
\vspace{0.5em}

\noindent
\begin{center}

\begin{minipage}{4.5cm}
  \begin{tabular}{|p{4.5em}|p{5.0em}|p{5.5em}|}
 \hline
 \hspace{1.7em} $x$   &  $y=g(x)$  & \hspace{1.5em}$(x,y)$ \\
            &  $\phantom{y}=x^2+3$   &  \\
 \hline
        \rule{0in}{2.0em}   &   &  \\ \hline
        \rule{0in}{2.0em}   &   &  \\ \hline
        \rule{0in}{2.0em}   &   &  \\ \hline
        \rule{0in}{2.0em}   &   &  \\ \hline
        \rule{0in}{2.0em}   &   &  \\ \hline
        \rule{0in}{2.0em}   &   &  \\ \hline
        \rule{0in}{2.0em}   &   &  \\ \hline
  \end{tabular}
\end{minipage}
\hspace{1.75in}
\begin{minipage}{.35\linewidth}
  \centering
  \cartX{-4}{4}{-1}{12}[0.7][{\draw[<->, color=red, thick, domain=-3.1:3.1, samples=25,  line cap=round]
            plot (\x, {\x *\x });}]
\end{minipage}%
\end{center}

\vfill








\example
The graph of $f(x)=x^2$ is given below.  Fill in the table and graph the
function $h(x) = x^2 - 3$.
\vspace{0.5em}

\noindent
\begin{center}

\begin{minipage}{4.5cm}
  \begin{tabular}{|p{4.5em}|p{5.0em}|p{5.5em}|}
 \hline
 \hspace{1.7em} $x$   &  $y=h(x)$  & \hspace{1.5em}$(x,y)$ \\
            &  $\phantom{y}=x^2-3$   &  \\
 \hline
        \rule{0in}{2.0em}   &   &  \\ \hline
        \rule{0in}{2.0em}   &   &  \\ \hline
        \rule{0in}{2.0em}   &   &  \\ \hline
        \rule{0in}{2.0em}   &   &  \\ \hline
        \rule{0in}{2.0em}   &   &  \\ \hline
        \rule{0in}{2.0em}   &   &  \\ \hline
        \rule{0in}{2.0em}   &   &  \\ \hline
  \end{tabular}
\end{minipage}
\hspace{1.75in}
\begin{minipage}{.35\linewidth}
  \centering
  \cartX{-4}{4}{-5}{10}[0.7][{\draw[<->, color=red, thick, domain=-3.1:3.1, samples=25,  line cap=round]
            plot (\x, {\x *\x });}]
\end{minipage}%
\end{center}

\vfill


\subsection{Horizontal Shift}

\example
The graph of $f(x)=x^2$ is given below.  Fill in the table and graph the
function $g(x) = (x-3)^2$.
\vspace{0.5em}

\noindent
\hspace{2mm}
\begin{minipage}{4.5cm}
  \begin{tabular}{|p{2.5em}|p{5.5em}|p{5.5em}|}
 \hline
 \hspace{1.0em} $x$   &  $y=g(x)$  & \hspace{1.5em}$(x,y)$ \\
            &  $\phantom{y}=(x-3)^2$   &  \\
 \hline
        \rule{0in}{2.0em}   &   &  \\ \hline
        \rule{0in}{2.0em}   &   &  \\ \hline
        \rule{0in}{2.0em}   &   &  \\ \hline
        \rule{0in}{2.0em}   &   &  \\ \hline
        \rule{0in}{2.0em}   &   &  \\ \hline
        \rule{0in}{2.0em}   &   &  \\ \hline
        \rule{0in}{2.0em}   &   &  \\ \hline
  \end{tabular}
\end{minipage}
\hspace{1.10in}
\begin{minipage}{.35\linewidth}
  \centering
  \cartX{-7}{7}{-1}{10}[0.7][{\draw[<->, color=red, thick, domain=-3.1:3.1, samples=25,  line cap=round]
            plot (\x, {\x *\x });}]
\end{minipage}%


\vfill


\example
The graph of $f(x)=x^2$ is given below.  Fill in the table and graph the
function $h(x) = (x+4)^2$.
\vspace{0.5em}

\noindent
\hspace{2mm}
\begin{minipage}{4.5cm}
  \begin{tabular}{|p{2.5em}|p{5.5em}|p{5.5em}|}
 \hline
 \hspace{1.0em} $x$   &  $y=h(x)$  & \hspace{1.5em}$(x,y)$ \\
            &  $\phantom{y}=(x+4)^2$   &  \\
 \hline
        \rule{0in}{2.0em}   &   &  \\ \hline
        \rule{0in}{2.0em}   &   &  \\ \hline
        \rule{0in}{2.0em}   &   &  \\ \hline
        \rule{0in}{2.0em}   &   &  \\ \hline
        \rule{0in}{2.0em}   &   &  \\ \hline
        \rule{0in}{2.0em}   &   &  \\ \hline
        \rule{0in}{2.0em}   &   &  \\ \hline
  \end{tabular}
\end{minipage}
\hspace{1.10in}
\begin{minipage}{.35\linewidth}
  \centering
  \cartX{-7}{7}{-1}{10}[0.7][{\draw[<->, color=red, thick, domain=-3.1:3.1, samples=25,  line cap=round]
            plot (\x, {\x *\x });}]
\end{minipage}%

\vfill


\subsection{Combining Vertical and Horizontal Shifts}

\example
The graph of $f(x)=x^2$ is given below.  Fill in the table and graph the
function $g(x) = (x+3)^2-5$.
\vspace{0.5em}

\noindent
\hspace{6mm}
\begin{minipage}{4.5cm}
  \begin{tabular}{|p{2.5em}|p{6.5em}|p{5.0em}|}
 \hline
 \hspace{0.5em} $x$   &  $y=g(x)$  & \hspace{1.5em}$(x,y)$ \\
            &  \hspace{0.15em} $=(x+3)^2-5$   &  \\
 \hline
        \rule{0in}{2.0em}   &   &  \\ \hline
        \rule{0in}{2.0em}   &   &  \\ \hline
        \rule{0in}{2.0em}   &   &  \\ \hline
        \rule{0in}{2.0em}   &   &  \\ \hline
        \rule{0in}{2.0em}   &   &  \\ \hline
        \rule{0in}{2.0em}   &   &  \\ \hline
        \rule{0in}{2.0em}   &   &  \\ \hline
  \end{tabular}
\end{minipage}
\hspace{1.20in}
\begin{minipage}{.35\linewidth}
  \centering
  \cartX{-7}{7}{-6}{10}[0.6][{\draw[<->, color=red, thick, domain=-3.1:3.1, samples=25,  line cap=round]
            plot (\x, {\x *\x });}]
\end{minipage}%

\vfill


\subsection{Vertical Stretch}

\example
The graph of $f(x)=\lvert x \rvert$ is given below.  Fill in the table and graph the
function $g(x) = 2 \lvert x \rvert$.
\vspace{0.5em}

\noindent
\hspace{6mm}
\begin{minipage}{4.5cm}
  \begin{tabular}{|p{2.5em}|p{5.5em}|p{5.0em}|}
 \hline
 \hspace{0.5em} $x$   &  $y=g(x)$  & \hspace{1.5em}$(x,y)$ \\
       &    $\phantom{y} = 2 \lvert x \rvert$   &  \\
 \hline
        \rule{0in}{2.0em}   &   &  \\ \hline
        \rule{0in}{2.0em}   &   &  \\ \hline
        \rule{0in}{2.0em}   &   &  \\ \hline
        \rule{0in}{2.0em}   &   &  \\ \hline
        \rule{0in}{2.0em}   &   &  \\ \hline
        \rule{0in}{2.0em}   &   &  \\ \hline
        \rule{0in}{2.0em}   &   &  \\ \hline
  \end{tabular}
\end{minipage}
\hspace{1.20in}
\begin{minipage}{.35\linewidth}
  \centering
  \cartX{-7}{7}{-1}{10}[0.6][{\draw[->, color=red, thick, domain=0:6.1, samples=25,  line cap=round]
            plot (\x, {\x }); \draw[<-, color=red, thick, domain=-6.1:0, samples=25,  line cap=round]
            plot (\x, {-\x });}]
\end{minipage}%

\vfill



\subsection{Vertical Compression}

\example
The graph of $f(x)=\lvert x \rvert$ is given below.  Fill in the table and graph the
function $h(x) = \frac{1}{2} \lvert x \rvert$.
\vspace{0.5em}

\noindent
\hspace{6mm}
\begin{minipage}{4.5cm}
  \begin{tabular}{|p{2.5em}|p{5.5em}|p{5.0em}|}
 \hline
 \hspace{0.5em} $x$   &  $y=h(x)$  & \hspace{1.5em}$(x,y)$ \\
       &    $\phantom{y} = \raisebox{-1.3ex}{\rule{0em}{1.7em}} \frac{1}{2} \lvert x \rvert$   &  \\
 \hline
        \rule{0in}{2.0em}   &   &  \\ \hline
        \rule{0in}{2.0em}   &   &  \\ \hline
        \rule{0in}{2.0em}   &   &  \\ \hline
        \rule{0in}{2.0em}   &   &  \\ \hline
        \rule{0in}{2.0em}   &   &  \\ \hline
        \rule{0in}{2.0em}   &   &  \\ \hline
        \rule{0in}{2.0em}   &   &  \\ \hline
  \end{tabular}
\end{minipage}
\hspace{1.20in}
\begin{minipage}{.35\linewidth}
  \centering
  \cartX{-7}{7}{-1}{10}[0.6][{\draw[->, color=red, thick, domain=0:6.1, samples=25,  line cap=round]
            plot (\x, {\x }); \draw[<-, color=red, thick, domain=-6.1:0, samples=25,  line cap=round]
            plot (\x, {-\x });}]
\end{minipage}%

\vfill


\subsection{Horizontal Compression and Stretching}

\example
The graph of $y=f(x)$ is given below.  Use this graph to sketch graph of
$y=f(2x)$ in the same space.
\vspace{0.5em}


{%
  \centering
\begin{tikzpicture}[yscale=2]
  %\draw[ystep=1cm,xstep=3.14159cm/2, cqcqcq,very thin] (0,-1) grid (3*3.1415,1);
  \draw[<->, black] (-1,0) -- (10,0);
  \draw[<->, black] (0, -1.2) -- (0, 1.2);
  \draw[-, color=red, thick, domain=0:9.42,samples=25,smooth,line cap=round]
            plot (\x, {sin(deg(\x) });
\end{tikzpicture}
}
\vfill


\newpage


\subsection{Reflection about the $x$-axis}

\example
The graph of $f(x)=x^2$ is given below.  Fill in the table and graph the
function $g(x) = -x^2$.
\vspace{0.5em}

\noindent
\begin{center}

\begin{minipage}{4.5cm}
  \begin{tabular}{|p{4.5em}|p{5.0em}|p{5.5em}|}
 \hline
 \hspace{1.7em} $x$   &  $y=g(x)$  & \hspace{1.5em}$(x,y)$ \\
            &  $\phantom{y}=-x^2$   &  \\
 \hline
        \rule{0in}{2.0em}   &   &  \\ \hline
        \rule{0in}{2.0em}   &   &  \\ \hline
        \rule{0in}{2.0em}   &   &  \\ \hline
        \rule{0in}{2.0em}   &   &  \\ \hline
        \rule{0in}{2.0em}   &   &  \\ \hline
        \rule{0in}{2.0em}   &   &  \\ \hline
        \rule{0in}{2.0em}   &   &  \\ \hline
  \end{tabular}
\end{minipage}
\hspace{1.75in}
\begin{minipage}{.35\linewidth}
  \centering
  \cartX{-4}{4}{-6}{6}[0.7][{\draw[<->, color=red, thick, domain=-2.2:2.2, samples=25,  line cap=round]
            plot (\x, {\x *\x });}]
\end{minipage}%
\end{center}

\vfill

\subsection{Reflection about the $y$-axis}

\example
The graph of $f(x)=\sqrt{x}$ is given below.  Fill in the table and graph the
function $h(x) = \sqrt{-x}$.
\vspace{0.5em}

\noindent
\begin{center}

\begin{minipage}{4.5cm}
  \begin{tabular}{|p{4.5em}|p{5.0em}|p{5.5em}|}
 \hline
 \hspace{1.7em} $x$   &  $y=h(x)$  & \hspace{1.5em}$(x,y)$ \\
            &  $\phantom{y}=\sqrt{-x}$   &  \\
 \hline
        \rule{0in}{2.0em}   &   &  \\ \hline
        \rule{0in}{2.0em}   &   &  \\ \hline
        \rule{0in}{2.0em}   &   &  \\ \hline
        \rule{0in}{2.0em}   &   &  \\ \hline
        \rule{0in}{2.0em}   &   &  \\ \hline
        \rule{0in}{2.0em}   &   &  \\ \hline
        \rule{0in}{2.0em}   &   &  \\ \hline
  \end{tabular}
\end{minipage}
\hspace{1.75in}
\begin{minipage}{.35\linewidth}
  \centering
  \cartX{-5}{5}{-5}{5}[0.7][{\draw[->, color=red, thick, domain=0:2.2, samples=25,  line cap=round]
            plot ({\x *\x }, \x);}]
\end{minipage}%
\end{center}

\vfill


\newpage


\subsection{Combining Transformations}


\example
The graph of $f(x)=\sqrt{x}$ is given below.  Fill in the table and graph the
function $g(x) = 2\sqrt{3-x } -4$.
\vspace{0.5em}

\noindent

\begin{minipage}{6.0cm}
  \begin{tabular}{|p{3.5em}|p{8.0em}|p{5.0em}|}
 \hline
 \hspace{1.7em} $x$   &  $y=g(x)$  & \hspace{1.5em}$(x,y)$ \\
            &  $\phantom{y}=2\sqrt{3-x } -4$   &  \\
 \hline
        \rule{0in}{2.0em}   &   &  \\ \hline
        \rule{0in}{2.0em}   &   &  \\ \hline
        \rule{0in}{2.0em}   &   &  \\ \hline
        \rule{0in}{2.0em}   &   &  \\ \hline
        \rule{0in}{2.0em}   &   &  \\ \hline
        \rule{0in}{2.0em}   &   &  \\ \hline
        \rule{0in}{2.0em}   &   &  \\ \hline
  \end{tabular}
\end{minipage}
\hspace{1.00in}
\begin{minipage}{.35\linewidth}
  \cartX{-8}{5}{-8}{5}[0.7][{\draw[->, color=red, thick, domain=0:2.2, samples=25,  line cap=round]
            plot ({\x *\x }, \x);}]
\end{minipage}
\vfill



\begin{goals}
  I know how to graph functions using Vertical and Horizontal transformations.  \\

  I know how to graph functions using Compressions and Stretches transformations.  \\
  I know how to graph functions using Reflections. \\
  I know how to graph functions when combining transformations. \\
\end{goals}




\end{document}
